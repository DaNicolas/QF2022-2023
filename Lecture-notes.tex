\documentclass[12pt, leqno, british]{amsart}
\usepackage[style=alphabetic, backend=biber]{biblatex}
\addbibresource{bibliography.bib}
\usepackage{a4, amsmath}
\usepackage{mathtools}
\usepackage{amssymb}
\usepackage{amsthm, amscd, mathdots}
\swapnumbers
\usepackage{enumerate}
\usepackage{hyperref}
\usepackage{cleveref}
\usepackage{csquotes}
\usepackage{color}
\usepackage{datetime}

\theoremstyle{definition}
\newtheorem{defi}{Definition}[subsection]
\theoremstyle{plain}
\newtheorem{prop}[defi]{Proposition}
\newtheorem{lem}[defi]{Lemma}
\newtheorem{thm}[defi]{Theorem}
\newtheorem{cor}[defi]{Corollary}
\theoremstyle{remark}
\newtheorem{rem}[defi]{Remark}
\newtheorem{eg}[defi]{Example}
\newtheorem{egs}[defi]{Examples}

\newcommand{\mc}{\mathcal}
\newcommand{\mf}{\mathfrak}
\newcommand{\mbb}{\mathbb}
\newcommand{\nat}{\mbb N}
\newcommand{\cc}{\mathbb C}
\newcommand{\qq}{\mbb Q}
\newcommand{\ovl}{\overline}

\DeclareMathOperator{\charac}{char}
\DeclareMathOperator{\id}{id}
\DeclareMathOperator{\Frac}{Frac}
\DeclareMathOperator{\Ker}{Ker}
\DeclareMathOperator{\Img}{Im}
\DeclareMathOperator{\Trd}{Trd}
\DeclareMathOperator{\Tr}{Tr}
\DeclareMathOperator{\Nrd}{Nrd}
\DeclareMathOperator{\GL}{GL}
\DeclareMathOperator{\Gal}{Gal}
\DeclareMathOperator{\ord}{ord}
\DeclareMathOperator{\trdeg}{trdeg}
\DeclareMathOperator{\supp}{supp}
\DeclareMathOperator{\rad}{rad}

\title{Quadratic forms and class fields II: lecture notes}
\author{Nicolas Daans}
\date{\today}
\address{Charles University, Faculty of Mathematics and Physics, Department of Algebra, Sokolov\-sk\' a 83, 18600 Praha~8, Czech Republic.}
\email{nicolas.daans@matfyz.cuni.cz}

\makeindex
\begin{document}
\maketitle
\tableofcontents

\subsection*{Notations and conventions}
We denote by $\nat$ the set of natural numbers.
We write $\nat^+$ for the proper subset of non-zero numbers.
For a ring $R$, we denote by $R^\times$ the set of invertible elements of $R$; if $R$ is a field, then $R^\times = R \setminus \lbrace 0 \rbrace$.

\section{Lecture 1}
We follow to a large extent the exposition from Lam's book \autocite{Lam}.
For this introductory course, we focus on fields of characteristic different from $2$.
We recommend the book of Elman, Karpenko and Merkurjev \autocite{ElmanKarpenkoMerkurjev} for those who want to learn more about quadratic form theory over fields of arbitrary characteristic.
\subsection{Bilinear and quadratic forms}
Let always $K$ be a field, $n \in \nat$.
\begin{defi}
A \emph{symmetric bilinear space over $K$}\index{symmetric bilinear!space} is a pair $(V, B)$ where
\begin{itemize}
\item $V$ is a finite-dimensional vector space over $K$, and
\item $B : V \times V \to K$ is a symmetric and bilinear map, i.e.~for all $x, x', y \in V$ and $a \in K$ we have
\begin{align*}
B(x, y) &= B(y, x), \\
B(x+x',y) &= B(x, y) + B(x', y), \\
B(ax, y) &= aB(x, y).
\end{align*}
\end{itemize}
We call the map $B$ a \emph{symmetric bilinear form on $V$}.\index{symmetric bilinear!form}
We define the dimension of $(V, B)$ to be the dimension of $V$, and denote this by $\dim(V, B)$ or simply $\dim B$.

Let $n = \dim (V, B)$. Given a basis $\mc{B} = (e_1, \ldots, e_n)$ of $V$, we define $\mc{M}_\mc{B}(B) = [B(e_i, e_j)]_{i, j=1}^n$, which we call \emph{the matrix of $(V, B)$ with respect to $\mc{B}$}.
\end{defi}
\begin{prop}
Let $V = K^n$ and let $\mf{B} = (e_1, \ldots, e_n)$ be the canonical basis.
Let $B$ be a symmetric bilinear form on $V$.
For column vectors $x = [x_1 \ldots x_n]^T$ and $y = [y_1, \ldots, y_n]^T$ we have
$$ B(x, y) = x^T\mc{M}_{\mc{B}}(B)y.$$
\end{prop}
\begin{proof}
This is clear from the bilinearity of $B$.
\end{proof}
\begin{defi}\label{D:QF}
A \emph{quadratic space over $K$}\index{quadratic!space} is a pair $(V, q)$ where
\begin{itemize}
\item $V$ is a finite-dimensional vector space over $K$, and
\item $q : V \to K$ is a map satisfying the following:
\begin{enumerate}
\item $\forall a \in K, \forall x \in V : q(ax) = a^2q(x)$,
\item the map $$\mf{b}_q : V \times V \to K : (x, y) \mapsto q(x+y) - q(x) - q(y) $$
is a symmetric bilinear form on $V$.
\end{enumerate}
\end{itemize}
We call the map $q$ a \emph{quadratic form on $V$}\index{quadratic!form}, and $\mf{b}_q$ its \emph{polar form}\index{polar form}.
We define the dimension of $(V, q)$ to be the dimension of $V$, and denote this by $\dim(V, q)$ or simply $\dim q$.
\end{defi}
\begin{defi}
Let $(V, B)$ and $(V', B')$ be symmetric bilinear spaces over $K$.
An isomorphism of $K$-vector spaces $I : V \to V'$ is called an \emph{isometry between $(V, B)$ and $(V', B')$}\index{isometry} if, for all $v, w \in V$, one has $B(v, w) = B'(I(v), I(w))$.
Similarly, given quadratic spaces $(V, q)$ and $(V', q')$ over $K$, an isomorphism of $K$-vector spaces $I : V \to V'$ is called an \emph{isometry between $(V, q)$ and $(V', q')$} if, for all $v \in V$, one has $q(v) = q'(I(v))$.

We call two symmetric bilinear spaces $(V, B)$ and $(V, B')$ (respectively two quadratic spaces $(V, q)$ and $(V', q')$) \emph{isometric}, which we denote by $(V, B) \cong (V', B')$ (respectively $(V, q) \cong (V', q')$) if there exists an isometry between them.
\end{defi}

Traditionally, a quadratic form over $K$ is often defined to be a homogeneous polynomial of degree $2$ over $K$.
\Cref{D:QF} can be seen as a coordinate-free version of this, as the following proposition indicates.
\begin{prop}\label{P:QF-coordinates}
Let $n \in \nat$ and let $f \in K[X_1, \ldots, X_n]$ be a homogeneous polynomial of degree $2$. The map
$$ q_f : K^n \to K : (x_1, \ldots, x_n) \mapsto f(x_1, \ldots, x_n)$$
is a quadratic form on $K^n$.

Conversely, given a quadratic space $(V, q)$ of dimension $n$, there exists a homogeneous degree $2$ polynomial $f \in K[X_1, \ldots, X_n]$ such that $(V, q)$ is isometric to $(K^n, q_f)$.
\end{prop}
\begin{proof}
For the first part of the statement, one verifies that the defined map satisfies the conditions stated in \Cref{D:QF}.

The second part of the statement is left as an exercise.
\end{proof}
\begin{prop}\label{P:isometric-coordinate}
Let $f, g \in K[X_1, \ldots, X_n]$ be homogeneous polynomials of degree $2$.
The quadratic spaces $(K^n, q_f)$ and $(K^n, q_g)$ are isometric if and only if there exists $C \in \GL_n(K)$ such that
$$ f\left(\left[ x_1 \ldots x_n \right]^T\right) = g\left((\left[ x_1 \ldots x_n\right] C)^T\right)$$
for all $x_1, \ldots, x_n \in K$.
\end{prop}
\begin{proof}
Exercise.
\end{proof}
\begin{eg}
Suppose $\charac(K) \neq 2$. Let $f(X_1, X_2) = X_1 \cdot X_2$ and $g(X_1, X_2) = X_1^2 - X_2^2$.
We observe that
$$ g\left(\frac{X_1 + X_2}{2}, \frac{X_1 - X_2}{2}\right) = f(X_1, X_2) $$
and thus, in view of \Cref{P:isometric-coordinate}, that $(K^2, q_f) \cong (K^2, q_g)$, with
\begin{displaymath}
C = \begin{bmatrix}
\frac{1}{2} & \frac{1}{2} \\ \frac{1}{2} & \frac{-1}{2}
\end{bmatrix}.
\end{displaymath}
\end{eg}
We saw that, to a quadratic form $q$, one can associate a symmetric bilinear form $\mf{b}_q$ on the same space.
It is also possible to obtain a quadratic form from a symmetric bilinear form: if $(V, B)$ is a symmetric bilinear space, then
$$ q_B : V \to K : v \mapsto B(v, v)$$
is easily seen to be a quadratic form.
If $\charac(K) \neq 2$, then these two operations are each others inverses (up to scaling by $\frac{1}{2}$), and hence the studies of quadratic and symmetric bilinear forms over $K$ are essentially the same:
\begin{prop}
Assume $\charac(K) \neq 2$. Let $(V, q)$ be a quadratic space.
Then $q$ is equal to the quadratic form associated to the form $\frac{1}{2}\mf{b}_q$.
Conversely, if $(V, B)$ is a symmetric bilinear space, then $B$ is equal to $\frac{1}{2}\mf{b}_q$ where $q = q_B$.
\end{prop}
\begin{proof}
This is a straightforward computation.
\end{proof}
Over fields of characteristic $2$, one can still associate to each quadratic form a symmetric bilinear form and to each symmetric bilinear form a quadratic form as before, but these operations are not invertible.
In fact, one needs to make an entirely separate study of quadratic and symmetric bilinear forms! We refer the interested reader to \autocite[Chapters I and II]{ElmanKarpenkoMerkurjev}.

We now go on to study basic properties of quadratic forms.
\begin{defi}\label{D:isotropic-represents-universal}
Let $(V, q)$ be a quadratic space over $K$.
\begin{itemize}
\item We call $q$ \emph{isotropic}\index{isotropic} if there exists $v \in V \setminus \lbrace 0 \rbrace$ such that $q(v) = 0$, or \emph{anisotropic}\index{anisotropic|see{isotropic}} otherwise.
\item Given $a \in K^\times$, we say that \emph{$q$ represents $a$}\index{representation!of an element by a form} if $\exists v \in V$ with $a = q(v)$.
We write
$$ D_K(q) = \lbrace a \in K^\times \mid \exists v \in V : a = q(v) \rbrace.$$
If $D_K(q) = K^\times$, we say that $q$ is \emph{universal}\index{universal}.
\end{itemize}
\end{defi}
\begin{egs}\label{E:hyp} \
\begin{enumerate}
\item Let $f(X_1, X_2) = X_1 \cdot X_2$. Then $q_f$ is isotropic, since $f(1, 0) = 0$.
$q_f$ is also universal, since, $f(1, a) = a$ for any $a \in K^\times$.
\item Let $f(X_1, X_2) = X_1^2 + X_2^2$.
$q_f$ is isotropic if and only if $-1$ is a square in $K$. $D_K(q_f)$ is the set of elements of $K$ which are a sum of two squares.
\item Let $f(X_1, X_2) = (X_1 + X_2)^2$. Then $q_f$ is isotropic since $f(1, -1) = 0$.
$D_K(q_f)$ consists of those elements of $K$ which are squares.
\end{enumerate}
\end{egs}
The last example is somewhat peculiar: the quadratic form $q_f$ with $f(X_1, X_2) = (X_1 + X_2)^2$ is of dimension $2$, but after a base change, one of the variables disappears. Indeed,
$$ f\left(X_1 - X_2, X_2\right) = X_1^2.$$
We will often want to exclude from our study quadratic forms which have this property.

For a $K$-vector space $V$, we denote by $V^\ast$ the dual space, i.e.~the space of linear maps $V \to K$.
Recall that $\dim(V^\ast) = \dim(V)$.
\begin{prop}\label{P:nonsingular-characterisations}
Let $(V, B)$ be a symmetric bilinear space.
Let $\mc{B}$ be a basis for $V$.
The following are equivalent.
\begin{enumerate}[(a)]
\item $\forall v \in V \setminus \lbrace 0 \rbrace$, $\exists w \in V$ : $B(v, w) \neq 0$,
\item The map $V \to V^\ast : v \mapsto (w \mapsto B(v, w))$ is a $K$-isomorphism.
\item The matrix $M_\mc{B}(B)$ is invertible.
\end{enumerate}
\end{prop}
\begin{proof}
Exercise.
\end{proof}
\begin{defi}
We call a symmetric bilinear space $(V, B)$ \emph{nonsingular}\index{nonsingular} if the above equivalent conditions hold.
We call a quadratic space $(V, q)$ nonsingular if its polar form is nonsingular.
Otherwise, we call the space \emph{singular}.
We use the same terminology for the symmetric bilinear and quadratic forms themselves.
\end{defi}
We now show that, at least over fields of characteristic not $2$, singular forms are precisely those for which, after a base change, one of the variables disappears.
\begin{prop}\label{P:nonsingular-polynomials}
Let $(V, q)$ be a quadratic space. Consider the statements
\begin{enumerate}[(a)]
\item\label{it:singular-1} $(V, q)$ is singular.
\item\label{it:singular-2} $\exists v \in V \setminus \lbrace 0 \rbrace$ such that for all $w \in V$ we have $q(w + v) = q(w)$.
\end{enumerate}
We have that \eqref{it:singular-2} $\Rightarrow$ \eqref{it:singular-1} in general.
If $\charac(K) \neq 2$, then \eqref{it:singular-1} and \eqref{it:singular-2} are equivalent.
\end{prop}
\begin{proof}
If \eqref{it:singular-2} holds, then $q(v) = q(v + 0) = q(0) = 0$, whence for any $w \in V$ we have $\mf{b}_q(v, w) = q(v + w) - q(v) - q(w) = 0$. Thus, $(V, q)$ is singular.

Now assume that $\charac(K) \neq 2$ and \eqref{it:singular-1} holds.
Then there exists $v \in V$ such that $\mf{b}_q(v, w) = 0$ for all $w \in V$.
But then in particular $0 = \mf{b}_q(v, v) = 2q(v)$ and thus $q(v) = 0$.
It follows that, for any $w \in V$, we have $q(v + w) = q(v) + q(w) + \mf{b}_q(v, w) = q(v)$, so \eqref{it:singular-2} holds.
\end{proof}
If $\charac(K) \neq 2$, a nonsingular quadratic form over $K$ is also called \emph{regular} or \emph{nondegenerate}.
Note that, if $\charac(K) = 2$, these terms have more specialised, distinct meanings.

\begin{rem}
So far, I have been somewhat careful in making the distinction between a symmetric bilinear/quadratic \textit{space} and a symmetric bilinear/quadratic \textit{form}.
This makes notation and speaking somewhat heavy. I will in the future often simply refer to the forms themselves, taking the convention that a symmetric bilinear/quadratic space `knows' its domain.
\end{rem}

\subsection{Orthogonality and diagonalisation}
\begin{defi}
Let $(V, B)$ be a symmetric bilinear space.
Let $v, w \in V$.
We say that $v$ and $w$ are \emph{orthogonal (with respect to $B$)}\index{orthogonal} if $B(v, w) = 0$.
We write $v \perp w$.

Let $v \in V$ and $M \subseteq V$.
We say that $v$ is \emph{orthogonal to $M$ (with respect to $B$)} if $B(v, w) = 0$ for all $w \in M$.
We write $v \perp M$.
Similarly, given $M' \subseteq V$, we say that $M$ is \emph{orthogonal to $M'$ (with respect to $B$)} if $B(v, w) = 0$ for all $v \in M$ and $w \in M'$, and write $M \perp M'$.

We write
$$ M^\perp = \lbrace v \in V \mid \forall w \in M : B(v, w) = 0 \rbrace$$
and call it the \emph{orthogonal space of $M$} - note that it is always a subspace of $V$. We write $v^\perp$ instead of $\lbrace v \rbrace^\perp$.

If $U \subseteq V$ is a subspace and $V = U \oplus U^\perp$, we call $U^\perp$ an \emph{orthogonal complement of $U$ in $V$}.
\end{defi}
Observe that a symmetric bilinear space $(V, B)$ is by definition nonsingular if and only if $V^\perp = \lbrace 0 \rbrace$.

\begin{prop}\label{P:dim-duality}
Let $(V, B)$ be nonsingular, $U \subseteq V$ a subspace. Then
$$ \dim U + \dim U^\perp = \dim V \quad\text{and}\quad (U^\perp)^\perp = U.$$
\end{prop}
\begin{proof}
Consider the $K$-linear maps
\begin{align*}
\varphi_1 &: U^\perp \to V^\ast : v \mapsto (w \mapsto B(v, w)) \\
\varphi_2 &: V^\ast \to U^\ast : f \mapsto f\vert_U.
\end{align*}
We observe that $\varphi_1$ is injective by the nonsingularity of $(V, B)$, that $\varphi_2$ is surjective, and that the image of $\varphi_1$ is precisely the kernel of $\varphi_2$ by definition of $U^\perp$.
As such, we compute that
\begin{align*}
\dim V &= \dim V^\ast = \dim(\Ker \varphi_2) + \dim(\Img \varphi_2) \\
&= \dim(\Img \varphi_1) + \dim U^\ast = \dim (U^\perp) + \dim(U)
\end{align*}
as desired.

For the second statement, observe that we trivially have $U \subseteq (U^\perp)^\perp$, but that, by the first claim, $\dim(U) = \dim((U^\perp)^\perp)$, whence $U = (U^\perp)^\perp$ as desired.
\end{proof}
We now define an operation on the set of quadratic spaces over $K$.
\begin{prop}\label{P:orth-sum}
Let $(V_1, q_1)$ and $(V_2, q_2)$ be quadratic spaces over $K$. Let $V = V_1 \times V_2$ and consider the map
$$ q : V \to K : (x, y) \mapsto q_1(x) + q_2(y). $$
Furthermore, consider the natural embeddings $\iota_1 : V_1 \to V : x \mapsto (x, 0)$ and $\iota_2 : V_2 \to V : x \mapsto (0, x)$.
We have that $(V, q)$ is a quadratic space, and $q$ is nonsingular if and only if both $q_1$ and $q_2$ are.
Furthermore, $i_1(V_1) \perp i_2(V_2)$ with respect to $\mf{b}_q$.
\end{prop}
\begin{proof}
Easy verification.
\end{proof}
\begin{defi}
Let $(V_1, q_1)$ and $(V_2, q_2)$ be quadratic spaces over $K$. We call the space $(V, q)$ defined in \Cref{P:orth-sum} the \emph{orthogonal sum of $(V_1, q_1)$ and $(V_2, q_2)$} and we denote the form $q$ by $q_1 \perp q_2$.
\end{defi}
\begin{prop}
Let $(V_i, q_i)$ and $(V_i', q_i')$ be quadratic spaces for $i = 1, 2, 3$. We have the following computation rules:
\begin{itemize}
\item $\dim(q_1 \perp q_2) = \dim(q_1) + \dim(q_2)$.
\item $q_1 \perp q_2 \cong q_2 \perp q_1$, and $q_1 \perp (q_2 \perp q_3) \cong (q_1 \perp q_2) \perp q_3$.
\item If $q_1 \cong q_1'$ and $q_2 \cong q_2'$, then $q_1 \perp q_1' \cong q_2 \perp q_2'$.
\end{itemize}
\end{prop}
\begin{proof}
Easy verifications.
\end{proof}
\begin{prop}\label{P:intrinsic-orth-sum}
Let $(V, q)$, $(V_1, q_1)$ and $(V_2, q_2)$ be quadratic spaces over $K$.
Then $q \cong q_1 \perp q_2$ if and only if there are $K$-subspaces $W_1$ and $W_2$ of $V$ with $W_1 \perp W_2$ with respect to $\mf{b}_q$, $V = W_1 \oplus W_2$, and such that $(W_i, q\vert_{W_i}) \cong (V_i, q_i)$ for $i = 1, 2$.
\end{prop}
\begin{proof}
Suppose that $\iota$ is an isomorphism $q_1 \perp q_2 \to q$ and let $W_1$ and $W_2$ be the images under this isomorphism of $V_1 \times \lbrace 0 \rbrace$ and $\lbrace 0 \rbrace \times V_2$ respectively. One verifies easily that these are as desired.

Conversely, assume that $W_1$ and $W_2$ are subspaces of $V$ with $W_1 \perp W_2$, $V = W_1 \oplus W_2$, and such that $(W_i, q\vert_{W_i}) \cong (V_i, q_i)$ for $i = 1, 2$.
Without loss of generality, we may assume that $V_i = W_i$ and $q_i = q\vert_{W_i}$.
Let $\iota$ be the unique $K$-linear map $V \to V_1 \times V_2$ which maps a vector $w \in W_1$ to $(w, 0)$ and a vector $w \in W_2$ to $(0, w)$.
Clearly this is an isomorphism of $K$-vector spaces.
Consider an arbitrary vector in $V$, which we may write as $w_1 + w_2$ for $w_1 \in W_1$ and $w_2 \in W_2$.
Since $W_1 \perp W_2$, we have that $\mf{b}_q(w_1, w_2) = 0$. We compute that
\begin{align*}
q(w_1 + w_2) &= q(w_1) + q(w_2) + \mf{b}_q(w_1, w_2) = q(w_1) + q(w_2) \\
&= q_1(w_1) + q_2(w_2) = (q_1 \perp q_2)(w_1, w_2) = (q_1 \perp q_2)(\iota(w_1 + w_2)).
\end{align*}
Hence $\iota$ is the desired isometry.
\end{proof}

We now discuss a special class of quadratic forms called diagonal forms. As it will turn out, in characteristic different from $2$, every quadratic form is isometric to a diagonal form (see \Cref{C:diagonalisation}).
\begin{defi}
Let $a_1, \ldots, a_n \in K$. We denote by $\langle a_1, \ldots, a_n \rangle_K$ the quadratic form
$$ K^n \to K : (x_1, \ldots, x_n) \mapsto \sum_{i=1}^n a_ix_i^2.$$
We call such a form a \emph{diagonal form}\index{diagonal form}.
If the field $K$ is clear from the context we might simply write $\langle a_1, \ldots, a_n \rangle$ instead of $\langle a_1, \ldots, a_n \rangle_K$.
\end{defi}
Note that $\langle a_1, \ldots, a_n \rangle_K \cong \langle a_1 \rangle_K \perp \ldots \perp \langle a_n \rangle_K$.
\begin{prop}\label{P:diagforms-singular}
Let $n \in \nat$ and $a_1, \ldots, a_n \in K$, let $q = \langle a_1, \ldots, a_n \rangle$.
If $\charac(K) \neq 2$, then $q$ is singular if and only if $a_i = 0$ for some $i \in \lbrace 1, \ldots, n \rangle$.
If $\charac(K) = 2$, then $q$ is singular as soon as $n \geq 2$.
\end{prop}
\begin{proof}
Exercise.
\end{proof}
\begin{prop}\label{P:diagonalisation}
Assume $\charac(K) \neq 2$.
Let $(V, q)$ be a quadratic space over $K$, $d \in K^\times$.
Then $d \in D_K(q)$ if and only if $q \cong \langle d \rangle \perp (V', q')$ for some quadratic space $(V', q')$.
\end{prop}
\begin{proof}
Clearly $d = d \cdot 1^2 + q'(0) \in D_K(\langle d \rangle \perp (V', q'))$ for any quadratic space $(V', q')$.

Conversely, assume that $d \in D_K(q)$.
Let $W$ be any subspace of $V$ such that $V = V^\perp \oplus W$.
Then $(W, q\vert_W)$ is nonsingular, and $D_K(q\vert_W) = D_K(q)$; see \Cref{P:decomposition-totally-isotropic} later on.
We may thus restrict our quadratic form to $W$, and assume without loss of generality that $q$ is nonsingular.

Now take $v \in V$ with $q(v) = d$.
Set $U = v^\perp$.
We have $v \not\in v^\perp$ (since $\mf{b}_q(v, v) = 2d \neq 0$) and $\dim(U) = \dim(V) - 1$ by \Cref{P:dim-duality}, hence $V = Kv \oplus U$.
Clearly $q\vert_{Kv} \cong \langle d \rangle$, so $q \cong \langle d \rangle \perp (U, q\vert_U)$ in view of \Cref{P:intrinsic-orth-sum}.
\end{proof}
\begin{cor}\label{C:diagonalisation}
Assume $\charac(K) \neq 2$, let $(V, q)$ be a quadratic space over $K$ of dimension $n$.
Then there exist $a_1, \ldots, a_n \in K$ such that $q \cong \langle a_1, \ldots, a_n \rangle$.
\end{cor}
\begin{proof}
Apply \Cref{P:diagonalisation} inductively.
\end{proof}

\subsection{Exercises}
\begin{enumerate}
\item Complete the proofs of \Cref{P:QF-coordinates}, \Cref{P:isometric-coordinate}, \Cref{P:nonsingular-characterisations} and \Cref{P:diagforms-singular}.
\item Illustrate by an example that the implication \eqref{it:singular-1} $\Rightarrow$ \eqref{it:singular-2} in \Cref{P:nonsingular-polynomials} does not hold in general if $\charac(K) = 2$.
\item Consider the quadratic form on $\qq^3$ given by the following polynomial:
$$f(X_1, X_2, X_3) = 3X_1^2 + 6X_1X_2 + 3X_2^2 - X_2X_3.$$
Explicitly construct a diagonal quadratic form $q$ on $\qq^3$ such that $(\qq^3, q_f) \cong (\qq^3, q)$.
\end{enumerate}

\section{Lecture 2}
Let always $K$ be a field.
\begin{defi}
Let $(V, q)$ be a quadratic space.
If $W$ is a subspace of $V$, the quadratic space $(W, q\vert_W)$ is called a \emph{subform}\index{subform} of $(V, q)$.
By abuse of terminology, we will also call a quadratic space $(U, q')$ which is isometric to $(W, q\vert_W)$ for some subspace $W$ of $V$ a subform of $(V, q)$.
\end{defi}
In this lecture, we will get closer to a classification of quadratic spaces over a given field, by decomposing quadratic spaces as orthogonal sums of subspaces with specific properties.

\subsection{Isotropic, totally isotropic, and hyperbolic forms}

Recall from \Cref{D:isotropic-represents-universal} the definition of an isotropic quadratic form.
\begin{defi}
Let $(V, q)$ be a quadratic space.
We call $(V, q)$ \emph{totally isotropic}\index{totally isotropic} if $q(v) = 0$ for all $v \in V$.
If $W$ is a subspace of $V$, we call $W$ totally isotropic if $(W, q\vert_W)$ is totally isotropic.
\end{defi}
\begin{prop}\label{P:radical-residue}
Assume $\charac(K) \neq 2$. Let $(V, q)$ be a quadratic space.
Then the map
$$ \ovl{q} : V/V^\perp \to K : v \mapsto q(\ovl{v})$$
is a well-defined nonsingular quadratic form.
\end{prop}
\begin{proof}
The well-definedness follows from the fact that, for $v \in V$ and $w \in V^\perp$, one has $q(v + w) = q(v)$ by \Cref{P:nonsingular-polynomials}.
It is then easy to verify that the map is a quadratic form.

For the nonsingularity, consider $v \in V$ such that $\ovl{v} \neq 0$, i.e.~$v \not\in V^\perp$.
Then there exists $w \in V$ with $0 \neq \mf{b}_q(v, w) = \mf{b}_{\ovl{q}}(\ovl{v}, \ovl{w})$, whereby $\ovl{v} \not\in (V/V^\perp)^\perp$.
Hence $(V/V^\perp)^\perp = \emptyset$, and thus $(V/V^\perp, \ovl{q})$ is nonsingular.
\end{proof}

The following observation was already used implicitly in the proof of \Cref{P:diagonalisation}.
\begin{prop}\label{P:decomposition-totally-isotropic}
Assume $\charac(K) \neq 2$.
Let $(V, q)$ be a quadratic space.
Let $W$ be an orthogonal complement of $V^\perp$.
We have that $$(V, q) \cong (V^\perp, q\vert_{V^\perp}) \perp (W, q\vert_W),$$ that $(V^\perp, q\vert_{V^\perp})$ is totally isotropic, and that $(W, q\vert_W) \cong (V/V^\perp, \ovl{q})$.
\end{prop}
\begin{proof}
The first isometry is immediate form \Cref{P:intrinsic-orth-sum}.
The fact that $(V^\perp, q\vert_{V^\perp})$ is totally isotropic follows from \Cref{P:nonsingular-polynomials}.

Finally, consider the map
$$ \iota : W \to V/V^\perp : w \mapsto \ovl{w}.$$
Since $W \cap V^\perp = \lbrace 0 \rbrace$ we have that $\iota$ is injective, hence by comparing dimensions, $\iota$ is bijective.
Furthermore, by definition we have for $w \in W$ that $q(w) = \ovl{q}(\ovl{w}) = \ovl{q}(\iota(w))$.
Hence we have obtained the required isometry $(W, q\vert_W) \cong (V/V^\perp, \ovl{q})$.
\end{proof}
We can thus, in characteristic away from $2$, decompose any quadratic space into the orthogonal sum of a totally isotropic space and a nonsingular space, and thus decomposition is unique up to isometry.

We now want to study nonsingular isotropic forms. One-dimensional quadratic forms are either anisotropic or totally isotropic.
\begin{defi}
We call the quadratic form $(K^2, q_f)$ with $f(X_1, X_2) = X_1 \cdot X_2$ the \emph{hyperbolic plane over $K$}\index{hyperbolic!plane} and denote it by $\mbb{H}_K$.
\end{defi}
\begin{prop}\label{P:hyperbolic-plane}
Let $(V, q)$ be a nonsingular, isotropic quadratic space over $K$.
Then there is a subform of $(V, q)$ isometric to $\mbb{H}_K$.
\end{prop}
\begin{proof}
Let $v \in V \setminus \lbrace 0 \rbrace$ be such that $q(v) = 0$.
Since $(V, q)$ is nonsingular, there exists $w \in V$ such that $a = \mf{b}_q(v, w) \neq 0$.
Observe that $w \not\in Kv$, so that $W = Kv \oplus Kw$ is a $2$-dimensional subspace of $V$.
Consider the map
$$ \iota : K^2 \to W : (x, y) \mapsto xa^{-1}v + y(w - q(w)a^{-1}v).$$
Clearly this is a $K$-isomorphism of vector spaces.
We compute that, for $x, y \in K$, we have
\begin{align*}
q(\iota(x, y)) &= q(xa^{-1}v + y(w - q(w)a^{-1}v)) \\
&= a^{-2}(x-yq(w))^2 q(v) + y^2q(w) + \mf{b}_q(a^{-1}(x - yq(w))v, yw) \\
&= 0 + y^2q(w) + a^{-1}(x - yq(w))y\mf{b}_q(v, w) = xy.
\end{align*}
Hence $(W, q\vert_W) \cong \mbb{H}_K$.
\end{proof}
In particular, it follows from \Cref{P:hyperbolic-plane} that the hyperbolic plane is, up to isometry, the only two-dimensional nonsingular isotropic quadratic form over $K$.
We also obtain the following
\begin{cor}
Every nonsingular isotropic quadratic space is universal.
\end{cor}
\begin{proof}
We know from \Cref{E:hyp} that $\mbb{H}_K$ is universal.
But by \Cref{P:hyperbolic-plane} every nonsingular isotropic quadratic space contains $\mbb{H}_K$ as a subspace, hence is also universal.
\end{proof}
\begin{cor}\label{C:representation-theorem}
Let $(V, q)$ be a nonsingular quadratic space and $d \in K^\times$.
We have that $d \in D_K(q)$ if and only if $q \perp \langle d \rangle_K$ is isotropic.
\end{cor}
\begin{proof}
Exercise.
\end{proof}

\begin{prop}\label{P:splitting-off}
Let $(V, q)$ be a nonsingular quadratic space, $W$ a subspace of $V$, $W'$ an orthogonal complement of $W$.
If $(W, q\vert_W)$ is nonsingular, then
$(V, q) \cong (W, q\vert_W) \perp (W', q\vert_{W'}),$
and also $(W', q\vert_{W'})$ is nonsingular.
\end{prop}
\begin{proof}
TO DO %TODO
\end{proof}
\begin{prop}\label{P:hyperbolic-form}
Let $(V, q)$ be a nonsingular quadratic space, $n \in \nat$.
The following are equivalent.
\begin{enumerate}
\item\label{it:hyperbolic-form-1} $V$ contains a totally isotropic subspace of dimension $n$,
\item\label{it:hyperbolic-form-2} $V$ contains a subform isometric to
\begin{displaymath}
\underbrace{\mbb{H}_K \perp \ldots \perp \mbb{H}_K}_{n \text{ times}}.
\end{displaymath}
\end{enumerate}
\end{prop}
\begin{proof}
For $n = 0$ there is nothing to show, assume from now on that $n \geq 1$.

Assume \eqref{it:hyperbolic-form-2}. 
Then $V$ has subspaces $W_1, \ldots, W_n$ such that $W_i \perp W_j$ and $W_i \cap W_j = \lbrace 0 \rbrace$ for any $i \neq j$ and such that $(W_i, q\vert_{W_i}) \cong \mbb{H}_K$.
Let $w_i \in W_i \setminus \lbrace 0 \rbrace$ be such that $q(w_i) = 0$.
Then $Kw_1 \oplus \ldots \oplus Kw_n$ is an $n$-dimensional totally isotropic subspace of $V$.

Conversely, assume \eqref{it:hyperbolic-form-1}.
We argue via induction on $n$ - recall that the case $n = 0$ is covered, so we assume $n \geq 1$.
Let $W$ be a totally isotropic subspace of $V$ of dimension $n$ and let $v \in W \setminus \lbrace 0 \rbrace$.
By \Cref{P:hyperbolic-plane} there is a subspace $W'$ of $V$ such that $(W', q\vert_{W'}) \cong \mbb{H}_K$, and by \Cref{P:splitting-off} this implies that $(V, q) \cong \mbb{H}_K \perp (U, q\vert_U)$ for some orthogonal complement $U$ of $W'$.
Observe that $\dim(W \cap W') \leq 1$, since $(W', q\vert_{W'})$ is not totally isotropic.
Hence $\dim(W \cap U) \geq n-1$, whereby $(U, q\vert_U)$ contains a totally isotropic subspace of dimension $n-1$.
The statement now follows by the induction hypothesis.
\end{proof}
\begin{cor}\label{C:hyperbolic-form}
Let $(V, q)$ be a nonsingular quadratic space of dimension $2n$, where $n \in \nat$.
The following are equivalent.
\begin{enumerate}
\item $V$ contains a totally isotropic subspace of dimension $n$,
\item $$(V, q) \cong \underbrace{\mbb{H}_K \perp \ldots \perp \mbb{H}_K}_{n \text{ times}}.$$
\end{enumerate}
\end{cor}
\begin{defi}
We say that a quadratic nonsingular quadratic space of dimension $2n$ (for some $n \in \nat$) is \emph{hyperbolic}\index{hyperbolic!space} if it contains a totally isotropic subspace of dimension $n$.

Given a quadratic space $(V, q)$, we define the \emph{Witt index}\index{Witt index} of $(V, q)$ to be the maximal possible dimension of a totally isotropic subspace of $(V/V^\perp, \ovl{q})$.
\end{defi}

\begin{prop}
Let $(V, q)$ be a nonsingular quadratic space. Then $(V, q) \perp (V, -q)$ is hyperbolic.
\end{prop}
\begin{proof}
TO DO %TODO
\end{proof}

\subsection{Witt's Theorems}
We are now in a position to prove the two most important structure theorems on quadratic forms, named after Ernst Witt.
We will prove them, as Witt did in the 1930'ies, under the assumption that $\charac(K) \neq 2$.
Versions in arbitary characteristic exist and can be proven with extra assumptions and a lot more work, see \autocite[Section 8]{ElmanKarpenkoMerkurjev}.
\begin{lem}\label{L:O(q)-transitive}
Assume that $\charac(K) \neq 2$.
Let $(V, q)$ be a quadratic space, and let $v, w \in V$ be such that $q(x) = q(y) \neq 0$.
There exists an isometry $\tau: (V, q) \to (V, q)$ such that $\tau(x) = y$.
\end{lem}
\begin{proof}
One computes that $q(v + w) + q(v - w) = 4q(v) \neq 0$, so at least one of $q(v+w)$ and $q(v-w)$ is non-zero.
Replacing $w$ by $-w$ if necessary, we may assume that $q(v-w) \neq 0$.
Now consider the map
$$\tau : V \to V : u \mapsto u - \frac{\mf{b}_q(u, x-y)}{q(x-y)}(x-y).$$
One verifies that $\tau$ gives an isometry $(V, q) \to (V, q)$, and that $\tau(v) = w$, as desired; see Exercise \eqref{ex-reflections}.
\end{proof}

\begin{thm}[Witt Cancellation Theorem]
Assume $\charac(K) \neq 2$.
Let $(V, q)$, $(V_1, q_1)$ and $(V_2, q_2)$ be quadratic spaces.
If $(V, q) \perp (V_1, q_1) \cong (V, q) \perp (V_2, q_2)$, then $(V_1, q_1) \cong (V_2, q_2)$.
\end{thm}
\begin{proof}
TODO %TODO
\end{proof}

\begin{thm}[Witt Decomposition Theorem]
Assume $\charac(K) \neq 2$.
Let $(V, q)$ be a quadratic spaces.
There exist quadratic spaces $(V_t, q_t)$, $(V_h, q_h)$ and $(V_a, q_a)$ such that
$$ (V, q) \cong (V_t, q_t) \perp (V_h, q_h) \perp (V_a, q_a)$$ where
\begin{itemize}
\item $(V_t, q_t)$ is totally isotropic,
\item $(V_h, q_h)$ is hyperbolic (or zero),
\item $(V_a, q_a)$ is anisotropic.
\end{itemize}
Furthermore, each of these spaces is determined uniquely up to isometry by $(V, q)$.
\end{thm}
\begin{proof}
TODO %TODO
\end{proof}

\subsection{Exercises}
\begin{enumerate}
%\item For a quadratic space $(V, q)$, define the \emph{quadratic radical of $q$} as the set
%$$ \rad(q) = \lbrace v \in V^\perp \mid q(v) = 0 \rbrace.$$
%Note that, if $\charac(K) \neq 2$, then $\rad(q) = V^\perp$.
%Show that \Cref{P:radical-residue} does not hold as stated when $\charac(K) = 2$, but still holds with $\charac(K) = 2$ if one replaces $V^\perp$ by $\rad(q)$.
\item Complete the proof of \Cref{C:representation-theorem}.
\item\label{ex-reflections} Let $(V, q)$ be a quadratic space, and consider for $v \in V$ with $q(v) \neq 0$ the map
$$ \tau_v : V \to V : w \mapsto w - \frac{\mf{b}_q(w, v)}{q(v)}v.$$
Show the following for any $v \in V$ with $q(v) \neq 0$:
\begin{enumerate}
\item $\tau_v$ is an isometry $(V, q) \to (V, q)$,
\item $\tau_v(v) = -v$, and for $w \in v^\perp$ we have $\tau_v(w) = w$,
\item If $w \in V$ is such that $q(v) = q(w)$ and $q(v-w) \neq 0$, then $\tau_{v-w}(v) = w$.
\end{enumerate}
\item Show that the following are equivalent for a field $K$ with $\charac(K) \neq 2$:
\begin{enumerate}
\item Any two nonsingular quadratic spaces over $K$ of the same dimension are isometric.
\item Every element of $K$ is a square.
\end{enumerate}
\end{enumerate}

\printindex
\printbibliography
\end{document}